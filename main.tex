% Options for packages loaded elsewhere
\PassOptionsToPackage{unicode}{hyperref}
\PassOptionsToPackage{hyphens}{url}
%
\documentclass[
]{article}
\usepackage{lmodern}
\usepackage{amssymb,amsmath}
\usepackage{ifxetex,ifluatex}
\ifnum 0\ifxetex 1\fi\ifluatex 1\fi=0 % if pdftex
  \usepackage[T1]{fontenc}
  \usepackage[utf8]{inputenc}
  \usepackage{textcomp} % provide euro and other symbols
\else % if luatex or xetex
  \usepackage{unicode-math}
  \defaultfontfeatures{Scale=MatchLowercase}
  \defaultfontfeatures[\rmfamily]{Ligatures=TeX,Scale=1}
\fi
% Use upquote if available, for straight quotes in verbatim environments
\IfFileExists{upquote.sty}{\usepackage{upquote}}{}
\IfFileExists{microtype.sty}{% use microtype if available
  \usepackage[]{microtype}
  \UseMicrotypeSet[protrusion]{basicmath} % disable protrusion for tt fonts
}{}
\makeatletter
\@ifundefined{KOMAClassName}{% if non-KOMA class
  \IfFileExists{parskip.sty}{%
    \usepackage{parskip}
  }{% else
    \setlength{\parindent}{0pt}
    \setlength{\parskip}{6pt plus 2pt minus 1pt}}
}{% if KOMA class
  \KOMAoptions{parskip=half}}
\makeatother
\usepackage{xcolor}
\IfFileExists{xurl.sty}{\usepackage{xurl}}{} % add URL line breaks if available
\IfFileExists{bookmark.sty}{\usepackage{bookmark}}{\usepackage{hyperref}}
\hypersetup{
  pdftitle={M-FIT Statistical Analysis Plan},
  hidelinks,
  pdfcreator={LaTeX via pandoc}}
\urlstyle{same} % disable monospaced font for URLs
\usepackage[margin=1in]{geometry}
\usepackage{longtable,booktabs}
% Correct order of tables after \paragraph or \subparagraph
\usepackage{etoolbox}
\makeatletter
\patchcmd\longtable{\par}{\if@noskipsec\mbox{}\fi\par}{}{}
\makeatother
% Allow footnotes in longtable head/foot
\IfFileExists{footnotehyper.sty}{\usepackage{footnotehyper}}{\usepackage{footnote}}
\makesavenoteenv{longtable}
\usepackage{graphicx,grffile}
\makeatletter
\def\maxwidth{\ifdim\Gin@nat@width>\linewidth\linewidth\else\Gin@nat@width\fi}
\def\maxheight{\ifdim\Gin@nat@height>\textheight\textheight\else\Gin@nat@height\fi}
\makeatother
% Scale images if necessary, so that they will not overflow the page
% margins by default, and it is still possible to overwrite the defaults
% using explicit options in \includegraphics[width, height, ...]{}
\setkeys{Gin}{width=\maxwidth,height=\maxheight,keepaspectratio}
% Set default figure placement to htbp
\makeatletter
\def\fps@figure{htbp}
\makeatother
\setlength{\emergencystretch}{3em} % prevent overfull lines
\providecommand{\tightlist}{%
  \setlength{\itemsep}{0pt}\setlength{\parskip}{0pt}}
\setcounter{secnumdepth}{5}
\usepackage{booktabs}
\usepackage{blkarray}
\usepackage{pdflscape}
\usepackage{lastpage}
\usepackage[ruled,vlined]{algorithm2e}
\usepackage{vhistory}
\usepackage{titling}
\usepackage{amsmath}
\usepackage{mathtools}
\usepackage{tcolorbox}
\usepackage{fancyhdr}
\usepackage{booktabs}
\usepackage{longtable}
\usepackage{array}
\usepackage{multirow}
\usepackage{wrapfig}
\usepackage{float}
\usepackage{colortbl}
\usepackage{pdflscape}
\usepackage{tabu}
\usepackage{threeparttable}
\usepackage{threeparttablex}
\usepackage[normalem]{ulem}
\usepackage{makecell}
\usepackage{xcolor}
\usepackage{bbding}
\usepackage{hyperref}
\hypersetup{
    colorlinks=true,
    linkcolor=blue,
    filecolor=blue,
    urlcolor=blue,
}
\usepackage{setspace}
\onehalfspacing
\urlstyle{same}
\usepackage[backend=biber,style=nature]{biblatex} %Imports biblatex package
\addbibresource{main.bib} %Import the bibliography file


% Declare commands
% -----------------

% The current version and title of the document (keep headers consistent)
\newcommand{\currentVersion}{September 2021 - Version 0.3}
\newcommand{\currentTitle}{M-FIT Statistical Analysis Plan}

\DeclareMathOperator*{\argmax}{arg\,max}
\DeclareMathOperator*{\argmin}{arg\,min}
\DeclarePairedDelimiter\nint{\lfloor}{\rceil}
\newcommand*{\argminl}{\argmin\limits}
\newcommand*{\argmaxl}{\argmax\limits}

% Combine number and name reference with label
\newcommand*{\fullref}[1]{\hyperref[{#1}]{\autoref*{#1} \nameref*{#1}}} % One single link


\newcommand{\blandscape}{\begin{landscape}}
\newcommand{\elandscape}{\end{landscape}}

\pagestyle{fancy}
\fancyhf{}
\fancyhead[LO]{\currentTitle}
\fancyfoot[LO]{\currentVersion}
\fancyfoot[RO]{\thepage\ of \pageref{LastPage}}
\renewcommand{\headrulewidth}{0pt}
\renewcommand{\footrulewidth}{0pt}

\title{\currentTitle}
\usepackage{etoolbox}
\makeatletter
\providecommand{\subtitle}[1]{% add subtitle to \maketitle
  \apptocmd{\@title}{\par {\large #1 \par}}{}{}
}
\makeatother
\subtitle{}
\author{}
\date{\vspace{-2.5em}\currentVersion}

\begin{document}
\maketitle

{
  \setcounter{tocdepth}{2}
  \tableofcontents
}

\hypertarget{version-history}{%
  \section*{Version History}\label{version-history}}
\addcontentsline{toc}{section}{Version History}

\begin{table}[H]
  \renewcommand{\arraystretch}{1.5}
  \begin{center}
    \begin{tabular}{lllp{5cm}}
      \hline
      Version & Date       & Author & Description   \\ \hline
      0.1     & 2021-04-23 & MAJ    & First draft   \\
      0.2     & 2021-07-27 & JAT    & Revisions \\
      0.3     & 2021-09-23 & JAT    & Revisions \\
      \hline
    \end{tabular}
  \end{center}
\end{table}

\clearpage

\hypertarget{preface}{%
  \section*{Preface}\label{Preface}}

This statistical analysis plan (SAP) outlinse the data and procedures for analysing effectiveness of trial interventions from the protocol M-FIT: an adaptive, randomised trial examining the comparative effectiveness of structured exercise programs on fatigue experienced by adult patients receiving dialysis.

The following documents were reviewed when preparing this SAP:

- M-FIT Study Protocol version 0.9.7, 2 September 2021

\clearpage

\hypertarget{abbreviations}{%
  \section*{Abbreviations}\label{abbreviations}}
\addcontentsline{toc}{section}{Abbreviations}

\begin{table}[H]
  \renewcommand{\arraystretch}{1.5}
  \centering
  \begin{tabular}{p{0.2\textwidth}p{0.7\textwidth}}
    \toprule
    Abbreviation  & Definition                                                                                                       \\
    \midrule
    30STS         & 30-second sit-to-stand test                                                                                      \\
    AKTN          & Australasian Kidney Trials Network                                                                               \\
    CKD           & Chronic Kidney Disease                                                                                           \\
    FACIT-Fatigue & Functional Assessment of Chronic Illness Therapy – Fatigue                                                       \\
    HADS          & Hospital Anxiety and Depression Scale                                                                            \\
    HD            & Haemodialysis                                                                                                    \\
    PD            & Peritoneal dialysis                                                                                              \\
    M-FIT         & Structured exercise program to reduce fatigue in patients receiving dialysis: an adaptive trial                  \\
    PFS           & Pittsburgh Fatigability Scale                                                                                    \\
    POMA          & Performance Oriented Mobility Assessment                                                                         \\
    PROMIS        & Patient-Reported Outcomes Measurement Information System                                                         \\
    PROMIS-APSRA  & Patient-Reported Outcomes Measurement Information System - Ability to Participate in Social Roles and Activities \\
    QALY          & Quality of life years                                                                                            \\
    SAP           & Statistical analysis plan                                                                                        \\
    SONG          & Standardised Outcomes In Nephrology                                                                              \\
    SONG-HD       & Standardised Outcomes In Nephrology - Haemodialysis                                                              \\
    SONG-HD-F     & Standardised Outcomes In Nephrology - Haemodialysis - Fatigue                                                    \\
    \bottomrule
  \end{tabular}
\end{table}

\clearpage

\hypertarget{introduction}{%
  \section{Introduction}\label{Introduction}}


\hypertarget{background}{%
  \subsection{Background and rationale}\label{background}}

Fatigue is one of the most common and debilitating symptoms in patients on dialysis.
Patients with chronic kidney disease (CKD) have also explicitly identified non-pharmacological self-management of lifestyle interventions to improve symptoms as a top research priority.
Based on some evidence to indicate that regular exercise improves cardiovascular outcomes, physical activity and quality of life across all stages of CKD, the primary objective of the M-FIT (Structured exercise progra\underline{M} to reduce \underline{F}atigue \underline{I}n patients receiving dialysis: an adaptive \underline{T}rial) study is to determine whether home-based exercise can improve fatigue in patients on dialysis.

M-FIT is a multi-centre, adaptive randomised comparative effectiveness trial of four structured exercise programs in adult patients receiving dialysis.
The trial will compare exercise regimes delivered by exercise physiologists and a mobile-app that aim to alleviate fatigue in patients on dialysis.
Specifically, the trial will evaluate

\begin{itemize}\tightlist
  \item the comparative effectiveness of a structured exercise program on fatigue in patients receiving dialysis
  \item the effectiveness of each therapy relative to an attention-control intervention
\end{itemize}

The main stakeholder/decision maker is the Australasian Kidney Trials Network (AKTN), who will facilitate translating and disseminating trial findings into treatment guidelines.
The full stakeholder pool is diverse with a range of objectives.

The study is open-label; patients, clinicians and analysts will know who was allocated to which therapy.
However, trial interim results are confidential until the final analysis.

The primary analysis is based on the 13-item 52-point FACIT-Fatigue\footnote{FACIT-F (as opposed to FACIT-Fatigue) is a related 40-item measure that also assesses self-reported fatigue and its impact upon daily activities and function. M-FIT uses the 13-item scale.} (Functional Assessment of Chronic Illness Therapy – Fatigue) score that will track individual-level fatigue over time.
The score ranges from 0 to 52 and is reverse coded so that \underline{higher scores imply less fatigue}.
The intervention therapies comprise:

\begin{enumerate}\tightlist
  \item attention-control: non-exercise based activity (stretching)
  \item walking
  \item resistance-only
  \item resistance and aerobic
\end{enumerate}

The therapies are delivered via an initial face-to-face assessment with an exercise physiologist and are followed-up over 12 weeks with weekly telehealth visits in weeks 1 to 4, 6 and 8 and a final follow up session in week 12.
Additionally, there is direction from a mobile-app.

Data collection is in part entered by study personnel, but also provided directly by the participant since they respond to surveys via the mobile application.

The design is a Bayesian (group sequential) adaptive randomised trial with response adaptive randomisation.

\clearpage

\hypertarget{background}{%
  \subsection{Study Objectives}\label{objectives}}


\hypertarget{primary-objective}{
  \subsubsection{Primary Objective}{\label{primary-objective}}
}

To assess the comparative effectiveness of alternative structured exercise programs, co-designed with patients, on fatigue, as measured by FACIT-Fatigue, in adult patients receiving dialysis.

\hypertarget{secondary-objectives}{
  \subsubsection{Secondary Objectives}{\label{secondary-objectives}}
}

\begin{enumerate}
  \def\labelenumi{\arabic{enumi}.}
  \item To assess the comparative effectiveness of four structured exercise programs on additional measures of fatigue, quality of life, mood, life participation, frailty, changes in body composition, exercise capacity, neuromuscular fitness, balance, physical activity, sleep, exercise adherence, SONG core outcomes, and hospital admissions.
  \item To compare the cost-effectiveness of four structured exercise programs in terms of incremental cost, and incremental health outcomes (quality-adjusted life year (QALY) and clinically important difference in fatigue).
  \item To assess the impact, fidelity, facilitators, and barriers of implementing the exercise programs in patients receiving dialysis.
  \item To evaluate the influence (if any) of participant preferences for the assigned interventions on fatigue and exercise adherence.
  \item To assess hospital admissions, mortality, exercise-related injuries, and cardiovascular events for all participants.
\end{enumerate}

\clearpage

\hypertarget{study-design}{%
  \section{Study Design}\label{study-design}}

The following information is summarised here from the study protocol.
For full details, refer to the relevant section in the study protocol.

\hypertarget{overview}{
  \subsection{Overview}\label{overview}}

The study is a propsective, multi-centre, adaptive, randomised comparative effectiveness trial.
The design uses group sequential methods and Bayesian response-adaptive randomisation.


\hypertarget{target-population}{%
  \subsection{Target Population}\label{target-population}}

All patients on haemodialysis (HD) or peritoneal dialysis (PD) who can provide informed consent in English, will be invited to participate in the trial.
Patients who are new to dialysis or those who have received a kidney transplant will be excluded.
Patient-reported outcome measures form a substantial part of the assessment in the trial, and only English versions are used.
Therefore, the eligibility criteria ensure that participants can comprehend the questions and respond accurately.
To ensure safety of the participants in the trial, exercise professionals will screen eligible patients for their capability to carry out movements required in all four exercise arms.

\hypertarget{inclusion-criteria}{
  \subsubsection{Inclusion Criteria}\label{inclusion-criteria}
}

To be eligible to participate in this trial, the participant must satisfy the following criteria:

\begin{enumerate}
  \def\labelenumi{\arabic{enumi}.}
  \tightlist
  \item on maintenance haemodialysis or peritoneal dialysis (>3 months) with a life expectancy of $\ge$ 12 months.
  \item > 18 years of age
  \item able to provide informed consent
  \item able to speak, read and write English
  \item access to a smart phone or tablet
  \item physically capable of carrying out all three exercises at intensity level 1 at minimum (as assessed by the site exercise professional).
\end{enumerate}

\hypertarget{exclusion-criteria}{
  \subsubsection{Exclusion Criteria}\label{exclusion-criteria}
}

Participants are excluded from M-FIT if they meet any of the following criteria:

\begin{enumerate}
  \def\labelenumi{\arabic{enumi}.}
  \tightlist
  \item presence of known cardiovascular disease that places the participant at an unacceptable risk of untoward event occurring during exercise training (as deemed by treating physician)
  \item have received or are expected to receive a kidney transplant within 12 months
  \item currently meeting the physical activity guidelines (150 mins/week of moderate intensity aerobic (cardio) activity and 2 sessions/week of resistance training)
\end{enumerate}


\hypertarget{interventions}{
  \subsection{Interventions}\label{interventions}
}

The trial interventions consist of three different structured exercise programs: walking, resistance training, combination aerobic (cardio) and resistance training, and a single attention-placebo control group (stretching).
There are five levels of exercise prescription within the walking, resistance training, and aerobic and resistance training intervention arms.
At baseline, the administering exercise professional will select the level of exercise deemed most appropriate for the patient.
The control arm will be assigned a set of stretches to complete for 12 weeks.

All participants will be provided with a mobile application.
The mobile application gives the participant access to demonstration videos applicable to their allocated group, and trial outcome assessment questionnaires.
Participants will be asked to download the application to their device.
The site exercise professional and study coordinator will be able to help with set up and setting of the initial treatment arm and exercise intensity level.
Once a participant has finished the treatment portion of the trial (12 weeks) they will be able to access all treatment arms and intensity levels.

The complete intervention will consist of:

\begin{enumerate}
  \def\labelenumi{\arabic{enumi}.}
  \item At baseline: an assessment (90 mins) with an exercise professional to assign a suitable initial intensity level and demonstration of the exercises, provision of the app and instructions by the research team.
  \item Weeks 1 to 4: weekly check-up sessions (10 min telehealth or in person) with the exercise professional and possible adjustment of the intensity level based on clinical and personal factors to ensure appropriate exercise intensity.
  \item Weeks 5 to 8: fortnightly check-up sessions (10 min telehealth or in person) with the exercise professional and possible adjustment of the intensity level.
  \item Weeks 12: final assessment session (90 mins) which repeats baseline assessments to assess exercise capacity. At this final assessment the exercise professional will include motivational interview and adherence strategies to improve exercise adherence during the 12 to 36-week period.
  \item Week 36: follow-up session (90 mins) to assess exercise capacity.
\end{enumerate}

Participants’ adherence to the interventions will be assessed with a self-report measure within the M-FIT application.
Research coordinators for each site will monitor and document participants’ data entry on adherence on a weekly basis with a check-up call to participants who have missed two or more sessions.

For full details of what each individual intervention entails, refer to the study protocol, Sections 5.5 and 15.3.

\hypertarget{randomisation}{%
  \subsection{Randomisation}\label{randomisation}}

Eligible participants will be randomised by research staff via a proprietary randomisation system and their assignment transcribed to the RedCap patient database.
The randomisation system was developed, validated and will be hosted by University of Sydney and has secure access, full audit trial and redundancy.
Participants, clinical staff and analysts are unblinded to allocation.

Participants will initially be randomised in 1:1:1:1 allocation between the four intervention arms.
Following each interim analysis, the allocation ratios to intervention arms will be updated.
The target allocation to the attention-control arm will remain fixed as the recipricol of the number of arm still available in the trial, for example, if one arm is dropped following an interim, then the allocation to control would be 1/3.
For the exercise arms, the target allocation will be proportional to the posterior probability that the intervention is most effective at reducing fatigue at 12 weeks.

If an intervention's posterior probability of being best falls below a pre-specified threshold then the allocation to that intervention may be set to 0 and the intervention dropped from the trial.
If the probability exceeds a pre-specified threshold then that intervention may be declared superior and randomisation stopped.

Randomisation probabilities will be updated by the trial statistician directly after the completion of each interim analysis \textbf{(or after meeting with the DSMB?)}.
Section \ref{response-adaptive-randomisation} contains further details on the calculation used to compute the randomisation probabilities.

\hypertarget{blinding}{%
  \subsection{Blinding}\label{blinding}}

  Due to the nature of the intervention, the site investigators, AKTN, treating clinicians, patients and the analysis group will not be blinded to treatment allocation. 
  The trial steering committee, statistical analysis plan developers, and outcome assessors will be blinded. 

\hypertarget{sample-size}{%
  \subsection{Sample Size}\label{sample-size}}

The study sample size was selected on the basis of feasability and to target a mean difference of half a standard-deviation in FACIT-Fatigue scores.
Assuming a two-sample independent one-sided $t$-test of size 0.025 and an alternative hypothesis of a difference in means of $\Delta=0.5\sigma$ ($\sigma$ the standard deviation), a sample size of 85 per arm has power 0.9.
Ignoring multiplicities, this results in a total sample size of 340 across four arms.
Allowing for a conservative amount of drop-out, this total sample size was increased to 400.

Further evaluation of power and trial operating characteristics for the adaptive design were accomplished via simulations.
We undertook 2,000 simulations under each of the trial scenarios of no treatment effects, and alternative scenarios where any superior treatments had an effect size of $0.5\sigma$ on FACIT-Fatigue scores, with a maximum sample size of 400 and assumed 20\% probability of drop-out.
The target allocation to each arm was initially equal and updated by using Bayesian response adaptive randomisation.
The simulations assumed compound symmetric correlation of 0.25 between the longitudinal outcomes of a participant.

In the null scenario, using a superiority decision threshold of 0.98 and inferiority threshold of 0.02/(number of active arms - 1), the probability of deciding any arm superior was < 1\% and the probability of dropping each arm as inferior was approximately 10\%.
When one intervention was superior to all others by $0.5\sigma$, a superiority threshold of 0.98 resulted in a decision of superiority of the superior arm in 87\% of trials and a decision of inferiority for the superior arm in 0\% of trials.

\clearpage

\hypertarget{outcomes}{
  \section{Study Outcomes}\label{outcomes}}

A summary of the study outcomes and their measurement occasions are provided in Table \ref{tab:study-outcomes}.
For full details of each outcome refer to the relevant section.

\begin{table}[!ht]
\centering
\small
\begin{tabular}{lrrrrr}
\toprule
Outcome                                                  & \makecell{Baseline\\(Clinic)}   & \makecell{Week 4\\(Phone)}     & \makecell{Week 8\\(Phone)}     & \makecell{Week 12\\(Clinic)}    & \makecell{Week 36\\(Clinic)}    \\
\midrule
\multicolumn{6}{l}{\textbf{Primary outcome}}                                                                                      \\
\hspace{1em}\hyperref[primary-outcome]{FACIT-Fatigue}    & \Checkmark         & \Checkmark & \Checkmark & \Checkmark & \Checkmark \\
\midrule
\multicolumn{6}{l}{\textbf{Secondary outcomes}}                                                                                   \\
\hspace{0.5em}\textit{Patient-reported outcomes}         &                    &            &            &            &            \\
\hspace{1em}\hyperref[outcome:song-hd-f]{SONG-HD-F}      & \Checkmark         & \Checkmark & \Checkmark & \Checkmark & \Checkmark \\
\hspace{1em}\hyperref[outcome:pfs]{PFS}                  & \Checkmark         & \Checkmark & \Checkmark & \Checkmark & \Checkmark \\
\hspace{1em}\hyperref[outcome:eq5d5l]{EQ-5D-5L}          & \Checkmark         & \Checkmark & \Checkmark & \Checkmark & \Checkmark \\
\hspace{1em}\hyperref[outcome:hads]{HADS}                & \Checkmark         & \Checkmark & \Checkmark & \Checkmark & \Checkmark \\
\hspace{1em}\hyperref[outcome:promis-apsra]{PROMIS-APSRA}& \Checkmark         & \Checkmark & \Checkmark & \Checkmark & \Checkmark \\
\hspace{1em}\hyperref[outcome:msus]{mSUS}                &                    &            &            &            & \Checkmark \\
\midrule
\hspace{0.5em}\textit{Exercise-professional assessments} &                    &            &            &            &            \\
\hspace{1em}\hyperref[outcome:30sts]{30 sec sit to stand}& \Checkmark         &            &            & \Checkmark & \Checkmark \\
\hspace{1em}\hyperref[outcome:mwp]{Modified wall push-up}& \Checkmark         &            &            & \Checkmark & \Checkmark \\
\hspace{1em}\hyperref[outcome:tug]{Timed up-and-go}      & \Checkmark         &            &            & \Checkmark & \Checkmark \\
\hspace{1em}\hyperref[outcome:6mwt]{6 min walk test}     & \Checkmark         &            &            & \Checkmark & \Checkmark \\
\hspace{1em}\hyperref[outcome:act]{Arm curl test}        & \Checkmark         &            &            & \Checkmark & \Checkmark \\
\hspace{1em}\hyperref[outcome:hgs]{Hand grip strength}   & \Checkmark         &            &            & \Checkmark & \Checkmark \\
\hspace{1em}\hyperref[outcome:poma]{Tinetti POMA}        & \Checkmark         &            &            & \Checkmark & \Checkmark \\
\hspace{1em}\hyperref[outcome:ffi]{Fried Frailty Index}  & \Checkmark         &            &            & \Checkmark & \Checkmark \\
\hspace{1em}\hyperref[outcome:body]{BMI}                 & \Checkmark         &            &            & \Checkmark & \Checkmark \\
\hspace{1em}\hyperref[outcome:body]{Waist circumference} & \Checkmark         &            &            & \Checkmark & \Checkmark \\
\hspace{1em}\hyperref[outcome:actigraph]{Physical activity/sleep (actigraph)}                     
                                                         & \Checkmark         &            &            & \Checkmark & \Checkmark \\
\midrule
\hspace{0.5em}\textit{SONG core outcomes}                &                    &            &            &            &            \\
\hspace{1em}Vascular access function                     &                    &            &            &            & \Checkmark \\
\hspace{1em}Technique survival                           &                    &            &            &            & \Checkmark \\
\hspace{1em}Peritoneal dialysis infections               &                    &            &            &            & \Checkmark \\
\midrule
\hspace{0.5em}\textit{Adverse events of interest}        &                    &            &            &            &            \\
\hspace{1em}Mortality                                    &                    & \Checkmark & \Checkmark & \Checkmark & \Checkmark \\
\hspace{1em}Hospitalisations                             &                    & \Checkmark & \Checkmark & \Checkmark & \Checkmark \\
\hspace{1em}Cardiac events                               &                    & \Checkmark & \Checkmark & \Checkmark & \Checkmark \\
\hspace{1em}Falls                                        &                    & \Checkmark & \Checkmark & \Checkmark & \Checkmark \\
\hspace{1em}Exercise related injuries                    &                    & \Checkmark & \Checkmark & \Checkmark & \Checkmark \\
\midrule
\hspace{0.5em}\textit{Exercise adherence}                &                    &            &            &            &            \\
\hspace{1em}Pre-exercise safety questionnaire            & \Checkmark         & \Checkmark & \Checkmark & \Checkmark & \Checkmark \\
\hspace{1em}Post-exercise questionnaire                  &                    & \Checkmark & \Checkmark & \Checkmark &            \\
\bottomrule
\end{tabular}
\caption{Summary of study outcomes and their measurement occasions.}
\label{tab:study-outcomes}
\end{table}

\hypertarget{primary-outcome}{
  \subsection{Primary Outcome}\label{primary-outcome}}

The primary outcome is fatigue as measured by the \textbf{Functional Assessment of Chronic Illness Therapy - Fatigue Scale (FACIT-Fatigue)} \cites{yellen1997measuring}{cella2002fatigue} at 12-weeks post-randomisation.
FACIT-Fatigue is a 13-item Likert scale, with each Likert item scored from 0 to 4.
Level of fatigue is measured as the total score aggregated across the 13 items, ranging from 0 to 52.
Lower scores imply a higher level of fatigue.

FACIT-Fatigue scores will be measured at baseline visit (week 0) and study visits at weeks 4, 8, and 12.
The data must be collected within five days of the scheduled visit.
For FACIT-Fatigue example norms for males and females between ages 18 to 70+ see \cite{montan2018general}.
For an example in a dialysis population see \cite{wang2015psychometric}.

\hypertarget{secondary-outcomes}{
  \subsection{Secondary Outcomes}\label{secondary-outcomes}}

\hypertarget{pro-outcomes}{
  \subsubsection{Patient-reported outcomes}\label{pro-outcomes}}

  \phantomsection
\label{outcome:song-hd-f}
The \textbf{Standardised Outcomes In Nephrology - Haemodialysis - Fatigue (SONG-HD-F)} \cite{ju2018establishing} is a 3 item Likert scale desgined to measure fatigue in the past week.
Each Likert item is a scale from 0 to 3 resulting in an aggregated score ranging from 0 to 9.

\phantomsection
\label{outcome:pfs}
The \textbf{Pittsburgh Fatigability Scale (PFS)} \cite{glynn2015pittsburgh} is a one-page, self-administered questionnarie that asks about level of exertion on a scale from 0 to 5 on 10 different activities.
For each activity, a response for both physical and mental fatigue are elicited.
A total score for phystical and mental fatigue are obtained by summing the respective rating for each activity, resulting in two scores each ranging from 0 to 50.

\phantomsection
\label{outcome:eq5d5l}
The \textbf{EuroQol 5 Dimension 5 Level (EQ-5D-5L)} is a standardised self-report survey measure of health-related quality of life that comprises five dimensions: mobility, self-care, usual activities, pain and discomfort, and anxiety and depression.
The five dimension each have five response levels of severity giving health scores ranging from 11111 (full health) to 55555 (worst health), equating to $5^5 = 3125$ potential health states.
For analysis, the responses are routinely converted into an index representing the utility of the health state.
The conversion, while not recommended by the EuroQol Group\footnote{EQ visual analog scale (VAS) is what EuroQol recommends for statistical analyses.}, proceeds by applying a societal preference function which generates scores anchored at 0 for death and 1 for perfect health and represent a societal valuation of health-related quality-of-life.
The conversion only works for complete data.
See \url{https://cran.r-project.org/web/packages/eq5d/vignettes/eq5d.html} and \url{https://link.springer.com/book/10.1007\%2F978-3-030-47622-9} for more detail.

\phantomsection
\label{outcome:hads}
The \textbf{Hospital Anxiety and Depression Scale (HADS)} \cite{zigmond1983hospital} is a questionnaire designed to measure anxiety and depression in a general medical population of patients.
It comprises seven questions each for anxiety and depression scored separately.
Each item is scored from 0 to 3 and aggregated across the seven questions resulting in two scores in the range of 0 to 21, one for anxiety and one for depression.

\phantomsection
\label{outcome:promis-apsra}
The \textbf{Patient Reported Outcomes Measurement Information System - Ability to Participate in Social Roles and Activities (PROMIS-APSRA) v2.0 4a} \cite{cella2019promis} is a 4 item questionnaire dessigned to assesses the perceived ability to perform one’s usual social roles and activities.
The 4 Likert items are on a scale of 1 to 5 resulting in aggregated raw scores ranging from 4 to 20.

\phantomsection
\label{outcome:msus}
The \textbf{modified System Usability Scale (mSUS)} assesses the the experience (user-friendliness, complexity, consistency, and user confidence) that users have with systems (applications, hardware) as well as benchmarking products against well-known, highly used systems.

\hypertarget{epa-outcomes}{
  \subsubsection{Exercise-physiologist assessed outcomes}\label{epa-outcomes}}

\phantomsection
\label{outcome:30sts}
The \textbf{30 sec sit to stand test (30STS)} \cites{rikli1999development}{jones199930}{macfarlane2006validity} is a test to assess lower body strength, endurance, and coordination.
The patient completes as many stands from a chair as possible in 30 seconds.

\phantomsection
\label{outcome:mwp}
The \textbf{modified wall push-up test (MWP)} is aimed at testing a patient’s upper body strength and endurance. 
The patient completes as many push-ups as possible in 30 seconds while standing against a wall. 

\phantomsection
\label{outcome:tug}
The \textbf{timed up-and-go test (TUG)} \cites{podsiadlo1991timed}{shumway2000predicting} is used to assess mobility, balance, walking ability, and fall risk.
Patients stand from a seated position, walk 3 meters, turn around, walk back, and sit back down. 
The aim is to complete this in as little time as possible.

\phantomsection
\label{outcome:act}
The \textbf{arm curl test (ACT)} \cites{rikli1999development}{liu2017predicting} is used as a predictor of elbow flexor strength. 
Patients complete as many arm curl repetitions as possible in 30 seconds using a designated dumbbell weight.

\phantomsection
\label{outcome:hgs}
The \textbf{hand-grip strength test (HGS)} \cite{liu2017predicting} is used as a measure of overall body strength.
Using a hand grip dynamometer, patients use one hand to apply as much force as possible in a squeezing action.
The amount of force is reported in kilograms.

\phantomsection
\label{outcome:6mwt}
The \textbf{six-minute walk test (6MWT)} \cites{butland1982two}{buvcar2016six} is a sub-maximal exercise test performed to assess aerobic endurance. 
For 6 minutes patients walk as far as possible utilising a 15 or 30-metre stretch of uninterrupted pathway as a course.
The total distance walked is reported (metres).

\phantomsection
\label{outcome:poma}
The \textbf{Tinetti Performance Oriented Mobility Assessment (POMA)} \cites{tinetti1986fall}{wyngaert2020associations} is a battery of assessments aimed at measuring a patients gait and balance abilities.
Scoring is done separately for balance (score range 0 to 12) and gait (score range 0 to 16), and the two results combined for a score out of 28.
A lower score implies higher risk of falls.

\phantomsection
\label{outcome:ffi}
The \textbf{Fried Frailty Index (FFI)} \cites{fried2001frailty}{het2015fried} is an assessment of physical functioning through both questions and physical assessments.
Five criteria comprise the assessment: shrinkage (weight loss), exhaustion, slowness, physical activity, and weakness.
Weight loss is either self-reported (or calculated if data is available) over the last year, exhaustion is determined by two questions from the Geriatric Depression Scale (GDS), physical activity is self-reported weekly activity, slowness is determined by timed up-and-go, and weakness is determined from hand-grip strength.
The sum of scores in these categories classifies individuals into three frailty conditions: not frail, pre-frail and frail. 

\phantomsection
\label{outcome:body}
Body compoisition will be assessed by \textbf{waist circumference} (cm) and \textbf{Body Mass Index (BMI)} (kg/m\textsuperscript{2}).

\phantomsection
\label{outcome:actigraph}
Minutes of \textbf{physical activity} and \textbf{sleep} will be measured using an Actigraph wrist accelerometer in the 7 days prior to intervention and the 7 days before intervention ends.

\hypertarget{song-outcomes}{
  \subsubsection{SONG-HD outcomes}\label{song-outcomes}}

\textbf{Vascular access function}

\textbf{Technique survival}

\textbf{Peritoneal dialysis infections}

\hypertarget{safety-outcomes}{
  \subsubsection{Safety outcomes}\label{safety-outcomes}}

\textbf{Hospitalisations} (admissions, LoS, days admitted).

\hypertarget{adherence-outcomes}{
  \subsubsection{Adherence}\label{adherence-outcomes}}
  
A \textbf{pre-exercise safety questionnaire} will be administered immediately before the patient begins every exercise session is used to determine the safety of the patient to complete an exercise session. The site exercise professional is also able to track any injuries, medication changes or health issues a patient may be having and adjust exercise prescription as necessary.
The questionnaire will ask about the participants capacity to exercise, and they self-report as limited capacity, then additional questions of their condition will be asked (cold/flu symptoms, nausea, muscle/joint/bone problems, etc.).

The \textbf{post-exercise questionnaire} is a simple report on patients’ self-adherence to the exercise regimen and intensity of exercise. This allows the site exercise professional to monitor patient performance and adjust exercise prescription where necessary to remain in the designated target intensity level.
The questionnaire will ask whether the participant completed the days exericse, and how difficult they perceived it to be.

\hypertarget{economic-outcomes}{
  \subsubsection{Economic outcomes}\label{economic-outcomes}}

Data linkage

Health-care resource use
  
Cost-effectiveness


\hypertarget{baseline}{
  \subsection{Baseline demographics}\label{baseline}}

The following information will be collected at baseline interview:

\begin{itemize}
  \tightlist
  \item age
  \item sex
  \item ethnicity
  \item country of birth
  \item primary kidney disease
  \item co-morbidities including smoking
  \item dialysis modality
  \item occupational status
  \item highest level of education achieved
  \item marital status
\end{itemize}

\hypertarget{subgroups}{%
  \subsection{Subgroups}\label{subgroups}}

Pre-specified subgroups are:

\begin{itemize}
  \item exercise preference
\end{itemize}

Refer to \fullref{subgroup-analyses} for details of planned subgroup analysis.

\clearpage

\hypertarget{statistical-modelling}{%
  \section{Statistical Modelling}\label{statistical-modelling}}

In this section we outline considerations relevant to the statistical modelling approach used in M-FIT.
These include a discussion on the analysis population, model specifications, subgroup, secondary and other analyses.

Inferences will be made within a Bayesian framework; general principles include:

\begin{itemize}
  \tightlist
  \item application of intention-to-treat
  \item accompanying descriptive statistics
  \item number of participants used in each analysis and reasons for exclusions
  \item comparisons across treatment groups with medians and 95\% highest density intervals
  \item accompanying assumptions for priors
  \item evaluation of goodness-of-fit
\end{itemize}

Due to repeated measure of outcomes during follow-up, analyses will generally use mixed models suitable to the outcome type.
For example, approximately continuous outcomes will be analysed using normal linear mixed effects models, counts using Poisson linear mixed models, and ordinal outcomes using cumulative logistic mixed models.

\hypertarget{analysis-population}{%
  \subsection{Analysis Population}\label{analysis-population}}

The target population relevant for the primary analysis is formed from Australian residents meeting the inclusion criteria, which nominally aligns to adults with end-stage kidney disease on  maintenance haemodialysis or peritoneal dialysis.
This population has significant potential for intercurrent events, but they are also a captive audience due to being critically dependent on dialysis for survival.
This suggests that adherence will likely be more of a problem than withdrawal.
Nevertheless, annual mortality in this group is around 15\% and there are high levels of  comorbidities, such as cardio-vascular disease, hypertension and diabetes.
These conditions may necessitate regular medicine and/or other therapies, which could be viewed as either underlying heterogeneity in the target population or a non-randomised modification of the therapies applied.
As such, there are a range of factors that may intermittently or permanently interfere with an individual's ability to continue with their allocated exercise prescriptions and these could conceivably cause issues in the analyses.
As an obvious example, there could easily be differential withdrawal or adherence across the treatment arms.

\subsubsection{Intention-to-treat Analyses}

Our primary analysis will include all participants randomised to an intervention arm regardless of whether intercurrent events disrupted the allocated exercise regime.
That is, we will adopt an intention-to-treat (ITT) principle for the analyses with all randomised participants contributing to the analysis such that:

\begin{itemize}
  \tightlist
  \item patients will be analysed in the group they were allocated to
  \item patients that do not receive (either partially or fully) the intervention will be retained
  \item protocol deviations will not result in automatic exclusion
  \item patients who die before end of the study  will be included in the analysis population until their time of death
\end{itemize}

\subsubsection{Per-Protocol Analyses}



\hypertarget{descriptive-statistics}{%
  \subsection{Descriptive statistics}\label{descriptive-statistics}}

Descriptive statistics will be calculated and presented by arm and in aggregate for all relevant baseline variables.
Categorical variables will be summarised by counts and proportions.
Continuous variables will be summarised by means and standard deviations or medians and interquartile ranges.

Descriptive and exploratory summaries all also be produced for study outcomes at each visit by arm and in aggregate.

For the primary outcome... MVN model for MAR summaries of mean and SD at each visit accounting for missingness.
Note this would be per-protocol summary (withdrawal implies non-adherence which may be different to those observed on the who adhered trial).

\hypertarget{primary-model}{%
  \subsection{Primary Outcome Model}\label{primary-model}}

The primary analysis will use a longitudinal model for participant fatigue as measured by FACIT-Fatigue over the first 12 weeks of follow-up.
All adaptations will be based on the relative intervention effects at 12 weeks post-randomisation, but we will also report the results at the other follow-up times.

While the FACIT-Fatigue score is ordinal in the range of 0-52, the primary analysis assumes a continuous response.
This approach simplifies both the implementation and interpretation.

The final analysis will occur at the maximum sample size (or lower if early stopping occurs) after all participants have reached the primary endpoint and the follow-up data collected to 36 weeks.
We will determine treatment comparative effectiveness based on the posterior distribution from the primary analysis.
The interim analyses will use the same model specification as the final analysis; also see \nameref{trial-adaptations-and-statistical-decisions}.

\hypertarget{model-specification}{%
  \subsubsection{Model specification}\label{model-specification}}

The pre-specified primary model  be a linear mixed model assuming an unstructured mean model for the outcomes at each visit occasion and treatment, and with subject-specific random intercept and visit coefficients.

We denote by $j\in\{0,1,2,3\}$ the visit occasion (0, 4, 8, or 12 weeks), and by $\texttt{trt}_i\in\{0,1,2,3\}$ the treatment assigned to participant $i$.
Denote by $y_{ij}$ the FACIT-Fatigue score for participant $i$ at visit $j$.
Then, for $i=1,...,n$ and $j=0,...,3$,
$$
\begin{aligned}
\mu_{ij}|\gamma_i &= (\alpha+\gamma_{i0}) + (\beta_j + \xi_{\texttt{trt}_i,j} + \gamma_{ij})1_{j>0} \\
y_{ij}|\mu_{ij},\sigma^2 &\sim \text{Normal}\left(\mu_{ij}, \sigma^2\right)
\end{aligned},
$$
where
$$
\begin{aligned}
  \gamma_i &\sim \text{Normal}\left(0, \Omega\right) \\
  \Omega &= \text{diag}(\tau)\Lambda\text{diag}(\tau).
\end{aligned}
$$

The model parameters are:

\begin{itemize}
  \tightlist
  \item $\alpha$ - the expected FACIT-Fatigue at baseline (shared across all intervention groups)
  \item $\beta_j$ - the change in FACIT-Fatigue from baseline to visit $j$ in the attention-control group
  \item $\xi_{\texttt{trt},j}$ - the effect of intervention $\texttt{trt}$ at visit $j$ relative to attention-control at visit $j$
  \item $\gamma_{i0}$ - random intercept for participant $i$
  \item $\gamma_{ij}$ - random coefficient for visit $j$ for participant $i$
  \item $\sigma^2$ - residual variance of FACIT-Fatigue
\end{itemize}
A landmark analysis at visit $j=3$ informs the primary comparison between treatments.

For priors, we specify
$$
  \begin{aligned}
  \sigma &\sim \text{Student-t}(3, 0, 5^2) \\
  \tau_j &\sim \text{Student-t}(3, 0, 5^2),\quad j=1,2,3 \\
  \Lambda &\sim \text{LKJ}(1) \\
  \alpha &\sim \text{Normal}\left(\text{med}(y), \text{mad}(y)^2\right) \\
  \text{(alt. [population norms]) } \alpha &\sim \text{Normal}(40, 5^2) \\
  \beta_j &\sim \text{Normal}\left(0, 5\right),\quad j=1,2,3 \\
  \xi_{\texttt{trt},j} &\sim \text{Normal}(0, 5),\quad \texttt{trt}=1,2,3,\quad j=1,2,3
\end{aligned}
$$
where $\text{med}(y)$ and $\text{mad}(y)$ are the sample median and scaled median absolute deviation respectively.

Although the unstructured mean model is the proposed default, a more parsimonious specification may be appropriate in light of the observed data (e.g. linear or piece-wise linear with respect to time since randomisation).

\hypertarget{subgroup-analyses}{%
  \subsection{Subgroup Analyses}\label{subgroup-analyses}}

Subgroup analyses for the primary model will examine heterogeneity in the FACIT-Fatigue score at 12 weeks post-randomisation arising from baseline exercise arm preference.
Participants state their preference at baseline and are categorised according to:

\begin{enumerate}\tightlist
  \item no particular preference
  \item preference for a particular exercise:
        \begin{enumerate}\tightlist
          \item preference for walking
          \item preference for resistance and aerobic
          \item preference for resistance-only
        \end{enumerate}
\end{enumerate}

Participants who have a preference may receive that intervention by chance when randomised.
This defines an additional two groups:
\begin{enumerate}\tightlist
  \item participants who received their preference
  \item participants did not receive their preference
\end{enumerate}

A participant having a preference is a baseline variable, and whether the participant receives their preferred exercise is post-randomisation.

A first subgroup analysis allows an expanded model where receiving the preferred exercise intervention may alter the effect of the intervention, but that any such effect is non-differential across the exercises.
The expanded model is then
$$
\begin{aligned}
\mu_{ij} = (\alpha + \gamma_{i0} + \zeta\texttt{hp}_i) + (\beta_j + \xi_{j}^\mathsf{T}\texttt{t}_i + \gamma_{ij} + \zeta\texttt{hp}_i + \kappa_j\texttt{rp}_i)1_{j>0}
\end{aligned}
$$
where here $\texttt{t}_i$ indicates treatment, $\texttt{hp}_i\in\{0,1\}$ indicates whether the participant expressed having a preference or not, and $\texttt{rp}_i\in\{0,1\}$ indicates whether the participant received their preference.
The model priors are analogous to the primary analysis model.

Alternatively, the effect of receiving preferred exericse may be allowed to vary according to the specific exercise preferred, by introducing an interaction term between $\texttt{t}_i$ and $\texttt{rp}_i$.

\hypertarget{secondary-analyses-primary}{%
  \subsection{Secondary Analyses of Primary Outcome}\label{secondary-analyses-primary}}

Ordinal model?

Item-specific analysis? E.g. instead of analysing aggregate FACIT-Fatigue, ordinal model of items themselves as factor?

\hypertarget{secondary-analyses}{%
  \subsection{Secondary Outcome Analyses}\label{secondary-outcome-analyses}}

The secondary outcomes are a mix of continuous, discrete and categorical variables, some collected at multiple timepoints over the duration of the study.
The following provides brief specifications for the analyses of each of the secondary outcomes.
The secondary outcomes are divided into groups for:

\begin{itemize}
  \item \nameref{clinical-outcomes}
  \item \nameref{patient-reported-outcomes}
  \item \nameref{exercise-professional-assessed-outcomes}
  \item \nameref{economic-outcomes}
\end{itemize}

\hypertarget{clinical-outcomes}{%
  \subsubsection{Clinical Outcomes}\label{clinical-outcomes}}

Standardised Outcomes in Nephrology (SONG) core outcomes follow.

\textbf{Vascular access function}

The aim is to compare the rate of vascular access repairs at 36 weeks by intervention.
Vascular access involves inserting a flexible tube into a blood vessel to provide a way of drawing blood/administering medicines over a period of weeks to years.

We will analyse the frequency of repairs using a mixed effects Poisson (Negative Binomial if overdispersion is apparent) model with fixed terms for treatment and random intercept for exercise physiologist.

\textbf{Technique survival}

TBD from workshop

\textbf{Peritoneal dialysis infections}

The aim is to compare the rates of PD associated infection at 36 weeks by intervention.
PD is a way to remove waste products from your blood when your kidneys are not able to and the procedures are associated with a high risk of infection of the peritoneum, subcutaneous tunnel and catheter exit site.
Typical rates of PD associated infection are around 0.24-1.66 episodes/patient/year.

We will analyse the frequency of infections using a mixed effects Poisson (Negative Binomial if overdispersion is apparent) model with fixed terms for treatment, timepoint and random intercept for exercise physiologist.

\textbf{Physical activity}

The aim is to use Actigraph/Fitbit data to compare the baseline adjusted duration of moderate and vigorous physical at 12 and 36 weeks by intervention.

We will analyse the durations using log-normal or gamma mixed effects models with fixed terms for treatment and timepoint, a random intercept for participant repeat measures and random intercept for exercise physiologist.

\textbf{Sleep}

The aim is to compare baseline adjusted duration of sleep (minutes) at 12 and 36 weeks by intervention.

We will analyse sleep duration with a log-normal or gamma mixed effects model with fixed terms for treatment and timepoint, a random intercept for participant repeat measures and random intercept for exercise physiologist.


\hypertarget{exercise-professional-assessed-outcomes}{%
  \subsubsection{Exercise professional assessed outcomes}\label{exercise-professional-assessed-outcomes}}



\textbf{Strength}

The aim is to use several supervised tests to compare baseline adjusted strength metrics at 12 and 36 weeks by intervention.
The tests are:

\begin{itemize}
  \item hand grip strength test (kg)
  \item timed up and go (seconds)
  \item 30 second sit to stand (frequency)
  \item modified wall push-up test (frequency)
  \item arm curl test (frequency)

\end{itemize}

We will analyse frequency of push-ups, sit-to-stand and arm curls using a Poisson (Negative Binomial if overdispersion is apparent) model with fixed terms for treatment and timepoint, a random intercept for participant repeat measures and random intercept for exercise physiologist.
We will analyse timed up and go and hand grip using a linear (or lognormal if deemed more appropriate) mixed effect model with fixed terms for treatment and timepoint, a random intercept for participant repeat measures and random intercept for exercise physiologist.

\textbf{Balance}

The aim is to use a supervised Tinetti balance test to compare baseline adjusted balance at 12 and 36 weeks by intervention.
The Tinetti Test (also known as Performance Oriented Mobility Assessment) comprises two sections, one examining static balance abilities in a chair and then standing, and the other gait.
Scoring is done on a three point ordinal scale with a range of 0 to 2.
The maximum total score is 28 points (12 for gait, 16 for balance; higher scores are better).

We will analyse the Tinetti score using linear mixed effect model with fixed terms for treatment and timepoint, a random intercept for participant repeat measures and random intercept for exercise physiologist.

\textbf{Cardiorespiratory}

The aim is to use the six-minute walk test (metres) to compare the baseline adjusted cardiorespiratory fitness at 12 and 36 weeks by intervention.

We will analyse distance covered during the six-minute walk test using log-normal mixed effect model for distance covered with fixed terms for treatment, timepoint, random intercept for participant repeat measures and random intercept for exercise physiologist.

TBD - Confirm if ``normal walking speed'' analysis required.

\textbf{Body composition}

The aim is to use the BMI to compare baseline adjusted body composition metrics at 12 and 36 weeks by intervention.
In order to compute body composition metrics, height (cm), weight (kg) and waist circumference (cm) will be collected.

We will analyse BMI using linear mixed effect model for BMI with fixed terms for treatment, timepoint, random intercept for participant repeat measures and random intercept for exercise physiologist.

\textbf{Frailty}

The aim is to use the Fried frailty index to compare patient baseline adjusted frailty at 12 and 36 weeks by intervention.
The Fried Frailty Index comprises five criteria for assessing weight loss, exhaustion, physical activity, slowness and weakness.
The sum of scores in the categories classifies individuals into three frailty conditions: not frail, pre-frail and frail.

We will analyse FFI using an ordinal mixed effects models for frailty with fixed terms for treatment, timepoint, random intercept for participant repeat measures and random intercept for exercise physiologist.

\textbf{Adherence}

The aim is to compare adherence in terms of responses provided from a pre and post exercise questionnaire at 12 and 36 weeks by intervention.

TBD - unclear as to what is required.

\textbf{Cognitive function}

TBD - unclear as to what is required.



\hypertarget{patient-reported-outcomes}{%
  \subsubsection{Patient reported outcomes}\label{patient-reported-outcomes}}


\textbf{Quality of Life}

The aim is to use the EQ-5D-5L score to compare baseline adjusted quality of life at 4, 8, 12 and 36 weeks by intervention.

We will analyse the EQ-5D-5L score using a linear (or beta) mixed effects model with fixed terms for treatment, timepoint, random intercept for participant repeat measures and random intercept for exercise physiologist.

TBD - intention for weights to convert into index for utility? Preferable to just report this data as summary statistics?


\textbf{SONG-HD Fatigue}

There are two aims for the secondary analysis on SONG-HD Fatigue:

\begin{enumerate}
  \item compare baseline adjusted fatigue at 4, 8, 12 and 36 weeks by intervention (preference are not part of this analysis) in patients receiving dialysis
  \item validation of the SONG-HD Fatigue score
\end{enumerate}

The SONG-HD Fatigue score comprises three items that assess
\begin{enumerate}
  \item the effect of fatigue on life participation
  \item tiredness
  \item level of energy
\end{enumerate}

These are measured on a four-point Likert scale and indicate severity, ranging from zero (not at all) to three (severely).
Given the limited range of the aggregate score (0-9 with higher scores implying greater fatigue)\footnote{Todo - confirm this is correct range.}, a numeric interpretation is not a suitable approximation.

We will analyse the score using an ordinal longitudinal mixed effects model parameterised per the primary analysis fixed terms for treatment, timepoint, random intercept for participant repeat measures and random intercept for exercise physiologist.

TBD - are the validation methods already planned out for the SONG-HD Fatigue score?

\textbf{Pittsburgh Fatigability}

The aim is to use the Pittsburgh Fatigability score will to compare the baseline adjusted fatigue at 4, 8, 12 and 36 weeks by intervention.
The Pittsburgh Fatigability Scale comprises ten questions relating to anticipated physical and mental fatigue under a range of activities with a score between 0 (no fatigue) to 5 (extreme fatigue) giving a total that ranges between 0 and 50.

We will analyse the score using linear mixed effects model with fixed terms for treatment, timepoint, random intercept for participant repeat measures and random intercept for exercise physiologist.

\textbf{Life participation}

The aim is to use the PROMIS Scale to compare baseline adjusted rates of social participation at 4, 8, 12 and 36 weeks by intervention.
The PROMIS (short form) scale is a 4-item tool that includes a component related to social participation.

TBD - unclear on details.

\textbf{Mood}

The aim is to use the Hospital Anxiety and Depression Scale to compare the baseline adjusted mood at 4, 8, 12 and 36 weeks by intervention.
HADS comprises 14 items; each item is coded 0 to 3 and there are seven relating to anxiety symptoms and seven relating to depression symptoms.
The scores for both anxiety and depression vary from 0 to 21 (higher scores imply worsening symptoms), depending on the presence and severity of the symptoms.

We will analyse the HADS score using separate linear mixed effects models for anxiety and depression with fixed terms for treatment, timepoint, random intercept for participant repeat measures and random intercept for exercise physiologist.




\hypertarget{economic-outcomes}{%
  \subsubsection{Economic outcomes}\label{economic-outcomes}}

\textbf{Health-care utilisation}

Evaluate cost-effectiveness of structured exercise program, in terms incremental cost, and incremental health outcomes (quality-adjusted life year (QALY) and clinically important difference in fatigues).

TBD

\textbf{Health-care cost/utility}

Cost-utility analysis (health care costs) at 36 weeks by intervention based on MBS/PBS use, data linkage, ANZDATA.

TBD



\hypertarget{qualitative-analyses}{%
  \subsection{Qualitative Analyses}\label{qualitative-analyses}}

Evaluation of the impact, fidelity, facilitators, and barriers of implementing the exercise program in patients receiving dialysis.
These analyses are beyond the scope of the SAP.

\hypertarget{missing-data}{%
  \subsection{Missing Data}\label{missing-data}}

Dialysis patients are generally not lost to follow up due to the nature of their treatment protocols.
Withdrawal of consent is also typically low in this population (around 5\%).
However, adherence can be a problem and safety is an ongoing concern.
Interim missingness might also occur if participants are on occassion unable or unwilling to complete the study questionnaires within the protocol timeframe.
Such interim missingness could be differential between treatments groups.
For example, if a particular exercise regimen does reduce fatigue, participants may be more willing to or capable of completing the requisite survey items compared 
to participants who are experiencing greater fatigue under a different intervention.

\subsubsection{Baseline}

Patterns of missingness for the baseline data will be reported by intervention group.
Baseline covariates which are to be included in the pre-specified models will be imputed, if missing, using information in the other baseline covariates.
Study outcomes measured at baseline will be included in the response model.

\subsubsection{Follow-up}

Patterns of missingness for follow-up data will be reported and summarised by intervention group across all measurement occasions.
We will investigate the relationships between baseline covariates and withdrawal or interim missingness of follow-up observations.

\begin{table}[!ht]
  \centering
  \small
  \begin{tabular}{lrrrr}
  \toprule
  N (\%) & Baseline & Week 4 & Week 8 & Week 12 \\
  \midrule
  xx (xx) & \Checkmark & \Checkmark & \Checkmark & \Checkmark \\
  xx (xx) & \Checkmark & \Checkmark & \Checkmark &   \\
  xx (xx) & \Checkmark & \Checkmark &  &   \\
  xx (xx) & \Checkmark &  &  &   \\
  xx (xx) &  &  \Checkmark & \Checkmark & \Checkmark   \\
  xx (xx) & \Checkmark &  &  \Checkmark &   \\
  \multicolumn{5}{c}{etc.} \\
  \bottomrule
  \end{tabular}
  \caption{Example summary of data missingness for study outcome. Observed values indicated by \Checkmark.}
\end{table}


Under the missing completely at random (MCAR) and missing at random (MAR) assumptions, there will be some loss of precision due to the missingness, but there is no bias in the parameter estimates when appropriate statistical methods are used.
However, under missing not at random (MNAR) the probability of missingness depends on the (unobserved) missing values.
When missingness is due to MNAR, there is both a loss in precision and bias and therefore sensitivity analyses are required.
MAR is commonly assumed, although MNAR is arguably more applicable for most settings.
For M-FIT, there is potential for MNAR as differential LTFU could be observed across treatment groups.

As missingness is expected to be relatively low, we will simply undertake analyses based on all available data.
If concerns arise regarding missingness (or a sensitivity analysis is requested) single value imputation may be used implementing a best-worst-case and worst-best-case sensitivity analyses to evaluate the potential scope of impact of missing values.
For continuous variables best-worst-case will use group means plus/minus two standard deviations.

\hypertarget{sensitivity-analyses}{%
  \subsection{Sensitivity Analyses}\label{sensitivity-analyses}}

TBD - can people add suggestions here?

- prior sensitivity anlayses
- more parsimonious model
- varying modelling assumptions (e.g. distributional, missing data etc.)


\hypertarget{safety-analyses}{%
  \subsection{Safety Analyses}\label{safety-analyses}}

The following safety data is of interest and recorded at 4, 8, 12 and 36 weeks (the Pre-exercise safety questionnaire is also collected at baseline).

\begin{itemize}
  \item Death
  \item Cardiac events
  \item Falls
  \item Exercise related injuries
  \item Pre-exercise safety questionnaire
\end{itemize}

Descriptive statistics of each safety variable will be tabulated in aggregate and by group and followup.

\clearpage

\hypertarget{statistical-quantities}{%
  \section{Statistical Quantities}\label{statistical-quantities}}

Statistical quantities obtained from the model parameters will be used to evaluate treatment effectiveness and direct the progression and adaptations for the trial.

In section \ref{primary-model} we introduced $\gamma_{\mathsf{time[i], trt[i]}}$ as the differences in the deviations from the standardised baseline FACIT-Fatigue score at each analysis.
Dropping the time indexing for notational brevity, at each analysis we compute the probability that arm $k$ is superior to all others $j$ by evaluating

\[
  \begin{aligned}
    \pi_{k} & = \mathsf{Pr}[\gamma_{k} - \max_{j\ne k} \gamma_j >0|\mathsf{data}] \quad \mathsf{for \  active} \quad k,j \in \{1, 2, 3, 4\}
  \end{aligned}
\]

which we will informally refer to as \textit{``p-best''} amongst the active arms.
Empirically, this can be calculated as the proportion of times that the $b^{th}$ posterior sample for $\widehat{\gamma}_k$ is the greatest across the active arms.

\hypertarget{treatment-superiority}{%
  \subsection{Treatment superiority}\label{treatment-superiority}}

Superiority of treatment arm $k$ relative to all other treatment arms is defined to occur when there is a very high probability that arm $k$ is the best.
Specifically, we define an arm to be superior when
\[
  \begin{aligned}
    \pi_{k} > \zeta_{\mathsf{sup}} \quad \mathsf{with} \quad \zeta_{\mathsf{sup}} = 0.98
  \end{aligned}
\]

where $\zeta_{\mathsf{sup}} = 0.98$ is the superiority decision threshold, identified through trial simulation.

\hypertarget{treatment-inferiority}{%
  \subsection{Treatment inferiority}\label{treatment-inferiority}}

Inferiority represents the situation where arm $k$ has a very low probability of being superior.
However, note that the existence of a superior arm does not imply that all others are inferior.
Inferiority of treatment arm $k$ is defined to occur when there is a very low probability that it is superior to all the others.
Specifically, we define an arm to be inferior when

\[
  \begin{aligned}
    \pi_{k} < (1 - \zeta_{\mathsf{sup}}) / (|\mathcal{A}_m| - 1)
  \end{aligned}
\]

where $|\mathcal{A}_m|$ is the number of currently active arms, i.e. we only include the active arms in the calculation.
Under this decision threshold, it is the case that when all arms bar one are inferior, the remaining arm must be superior.

\hypertarget{treatment-effectiveness}{%
  \subsection{Treatment effectiveness}\label{treatment-effectiveness}}

The pseudo control arm is used as a referent from which the effectiveness of the \textit{active} therapies can be evaluated.
Specifically, we define a therapy to be effective relative to the pseudo-control when

\[
  \begin{aligned}
    \pi_{k,1} & = \mathsf{Pr}[\gamma_{k} - \gamma_{1} >0|\mathsf{data}] > \epsilon
  \end{aligned}
\]

where $\pi_{k,1}$ denotes a comparison of arms $k = 2, 3, 4$ with arm $1$ that we assume indexes the control arm.

\clearpage

\hypertarget{trial-adaptations-and-statistical-decisions}{%
  \section{Trial Adaptations and Statistical Decisions}\label{trial-adaptations-and-statistical-decisions}}

As the trial proceeds, the accrued information is used to make decisions on the progression of the trial based on pre-specified adaptations and decision thresholds.
For example, treatment arms may show evidence of being inferior relative to all others and therefore enrollment would cease for these arms.

For adaptations internal to the trial, predefined rules are in place to inform trial decisions conditional on the primary model.
Pending independent oversight from a DSMC, these statistical decisions will inform conclusions such as deciding arms to be superior or inferior.
The following sections outline these adaptations.

\hypertarget{sequential-analyses}{%
  \subsection{Sequential Analyses}\label{sequential-analyses}}

We will run regular interim analyses scheduled according to the sample size of participants with follow-up to the primary endpoint.
The first analysis will occur when \(n=100\) participants have reached the 12 week primary endpoint (regardless of whether the outcome was observed or not).
After the first analysis, interim analyses will occur for every additional \(100\) participants who reach the primary endpoint up to the maximum sample size of \(400\).
If recruitment is slow, we may seek permission to revise this rule.

The analyses will include the data from all the enrolled participants at the time when the $'00^{th}$ participant reaches the 12 week primary endpoint.
That is to say that some of the participants will only contribute their baseline data, some will contribute their baseline and 4 week follow up and so on.

\hypertarget{stopping-rules}{%
  \subsection{Stopping rules}\label{stopping-rules}}

Enrollment into the trial may be stopped prior to the maximum sample size if:

\begin{itemize}\tightlist
  \item one of the exercise arms is declared superior and effective at the nominated decision thresholds
  \item all exercise arms have been declared inferior at the nominated decision threshold
\end{itemize}

Additionally, the trial may be stopped early due to safety concerns.
Otherwise the trial will continue until the maximum sample size.

\hypertarget{response-adaptive-randomisation}{%
  \subsection{Response-Adaptive Randomisation}\label{response-adaptive-randomisation}}

Target allocation to exercise arms will be updated to be proportional to a transformation of the probability an exercise arm is the best (in terms of reducing fatigue), the variance of the posterior change in fatigue divided, and the sample size on that arm.
Note that the response adaptive randomisation is based on the comparison amongst the active exercise arms at the week 12 follow up.

Following an interim analysis, an exercise arm may be dropped if:
\begin{itemize}\tightlist
  \item it is inferior to the other exercise arms (probability of it being superior below decision threshold)
  \item it is inferior to the attention-control arm (probability of it being effective below decision threshold).
\end{itemize}

Denote by $\mathcal{A}_t$ the set of exercise arm indices which will continue in the trial following the current interim $t$.
If an exercise arm satisfies either of the above criteria, then it may be removed from $\mathcal{A}_t$.

We will compute allocation probabilities according to
$$
\begin{aligned}
r_{t0} &= \frac{1}{|\mathcal{A}_t + 1} \\
r_{tk} &= \begin{cases}
  0 & \text{if } k \notin \mathcal{A}_t \\
  (1 - r_0) \frac{f(\pi_k, \mathbb V[\gamma_k|\mathsf{dat}], n_k)}{\sum_{j=1}^3 f(\pi_j, \mathbb V[\gamma_j|\mathsf{dat}], n_j)} & \text{if }k\in\mathcal{A}_t
\end{cases}, \quad k=1,2,3.
\end{aligned}
$$
where $f(\pi, V, n) = \sqrt{\pi V/(n + 1)}$, and $\mathcal{A}_0 = \{1,2,3\}$ is the avaialable exercise arms at trial start.
Once an arm is dropped from the trial, it is unavailable for the remainder of the trial, that is, $\mathcal{A}_{t+1} \subseteq \mathcal{A}_t$.

In practice, the target allocations may be updated prior to confirmation of the decision on arm dropping (e.g. TSC may have final decision in whether to drop an arm), in which case the relevant values of $r_t$ 
will be set to zero only after such a confirmation has been made, and the remaining arm allocations re-normalised to sum-to-one.

In essence, two updates would then occur.
The first being an update without any arm dropping
$$
\begin{aligned}
  r_{t0}^\star &= r_{t-1,0} \\
  r_{tk}^\star &= (1 - r_{t0}^\star)\frac{f(\pi_k, \mathbb V[\gamma_k|\mathsf{dat}], n_k)}{\sum_{k=1}^3 f(\pi_k, \mathbb V[\gamma_k|\mathsf{dat}], n_k)}, \quad k\in\mathcal{A}_{t-1}
\end{aligned}
$$
and then following confirmation that arms are to be dropped
$$
\begin{aligned}
  r_{t0} &= \frac{1}{|\mathcal{A}_t + 1} \\
  r_{tk} &= \begin{cases}
    0 & \text{if } k \notin \mathcal{A}_t \\
    (1 - r_{t0}) \frac{r_{tk}^\star}{\sum_{j\in\mathcal{A}_t} r_{tj}^\star} & \text{if } k \in \mathcal{A}_t.
  \end{cases}, \quad k=1,2,3.
\end{aligned}
$$
In this manner, if a decision is made to \emph{not} drop an arm despite it meeting the criteria, the update to allocations does not need to be revised.

\hypertarget{model-deviations}{%
  \subsection{Model Deviations}\label{model-deviations}}

The primary analysis model will be assessed for adequacy.
Additional models (either simpler or more complex) may be investigated as part of checks of sensitivity, stability, and model fit.
If any issues or concerns arise, all changes or updates to the specified primary model will be documented and reported including justification of the changes.

For example, if there are issues with the random coefficients model, then a saturated mean model with unstructured covariance may be used instead, ignoring subject specific trajectories.

\clearpage

\hypertarget{interim-analyses-and-trial-reporting}{%
  \section{Interim Analyses and Trial Reporting}\label{interim-analyses-and-trial-reporting}}

Following each scheduled interim analysis, a closed DSMC report will be generated with the following content:

\begin{itemize}\tightlist
  \item
        Executive summary
  \item
        Protocol synopsis
  \item
        Report overview
  \item
        Suggested communication to the study team investigators
  \item
        Enrollment status including rates of enrollment
  \item
        Participant status
  \item 
        Data completeness
  \item
        Descriptive statistics for baseline demographic, co-morbidities, and other baseline factors, stratified by treatment arm and in aggregate
  \item
        Descriptive statistics of adherence to treatment protocol, protocol violations and dropouts
  \item
        Safety data
  \item
        Protocol deviations
  \item 
        Descriptive statistics pertaining to the primary outcome
  \item
        Inference results from primary analysis
  \item
        Appendix: As required
\end{itemize}


\hypertarget{platform-conclusions}{%
  \subsection{Platform Conclusions}\label{platform-conclusions}}

The DSMC will be notified when any statistical trigger occurs and is collectively responsible for deciding whether to recommend public announcements to the TSC.

\hypertarget{trial-reporting}{%
  \subsection{Trial Reporting}\label{trial-reporting}}

Communication of trial outcomes will occur under the following conditions:

\begin{enumerate}
  \def\labelenumi{\arabic{enumi}.}
  \tightlist
  \item
        the maximum sample size has been reached
  \item
        all arms are deemed inferior to standard of care
\end{enumerate}

\textbf{IMPORTANT}: Other than the routine interim analyses, no reporting, nor other analyses will occur prior to the above criteria being met.
Exceptions may be made to this rule:

\begin{itemize}
  \tightlist
  \item to support necessary applications for funding
  \item based on requests from the DSMC for additional information
\end{itemize}

\clearpage

\printbibliography[heading=bibintoc]

\end{document}
